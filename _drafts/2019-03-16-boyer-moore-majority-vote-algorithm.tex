% Created 2019-05-04 Sat 23:12
% Intended LaTeX compiler: pdflatex
\documentclass[11pt]{article}
\usepackage[utf8]{inputenc}
\usepackage[T1]{fontenc}
\usepackage{graphicx}
\usepackage{grffile}
\usepackage{longtable}
\usepackage{wrapfig}
\usepackage{rotating}
\usepackage[normalem]{ulem}
\usepackage{amsmath}
\usepackage{textcomp}
\usepackage{amssymb}
\usepackage{capt-of}
\usepackage{hyperref}
\usepackage{ctex}
\author{Ynjxsjmh}
\date{\textit{<2019-03-16 Sat>}}
\title{摩尔投票算法及其变体}
\hypersetup{
 pdfauthor={Ynjxsjmh},
 pdftitle={摩尔投票算法及其变体},
 pdfkeywords={},
 pdfsubject={},
 pdfcreator={Emacs 26.1 (Org mode 9.1.9)}, 
 pdflang={English}}
\begin{document}

\maketitle
\tableofcontents



\section{波义尔摩尔投票算法(Boyer–Moore majority vote algorithm)}
\label{sec:orgc33ba11}
Boyer–Moore majority vote algorithm 是用来查找一个长度为n的序列中出现次数大于 \(\lfloor\frac{n}{2}\rfloor\) 中的数。该算法的时间复杂度是O(n),空间复杂度是O(1)。该算法有一个前提是必须存在这样一个数,否则算法不起作用。

\begin{verbatim}
int majorityElement(vector<int>& nums) {
    int candidate = 0;
    int count = 0;
    int n = nums.size();
    for(int i = 0; i < n; i++) {
        if (count == 0) {
            candidate = nums[i];
            count = 1;
        } else if (candidate == nums[i]) {
            count++;
        } else {
            count--;
        }
    }
    return candidate;
}
\end{verbatim}

\section{要求寻找出现次数大于 \(\lfloor\frac{n}{3}\rfloor\) 的数}
\label{sec:org29004f3}
当要求寻找出现次数大于 \(\lfloor\frac{n}{3}\rfloor\) 的数时,满足该条件的数最多有两个。

$$
n > (\frac{1}{3}n + 1) * 2 在 n >= 6 时成立
$$

\begin{verbatim}
vector<int> majorityElement(vector<int>& nums) {
    int count1 = 0, count2 = 0;
    int candidate1 = 0, candidate2 = 0;
    int n = nums.size();

    for (int i = 0; i < n; i++) {
        if (nums[i] == candidate1) {
            count1++;
        } else if (nums[i] == candidate2) {
            count2++;
        } else if (count1 == 0) {
            count1 = 1;
            candidate1 = nums[i];
        } else if (count2 == 0) {
            count2 = 1;
            candidate2 = nums[i];
        } else {
            count1--;
            count2--;
        }
    }

    vector<int> result;

    if (count(nums.begin(), nums.end(), candidate1) > n/3) {
        result.push_back(candidate1);
    }

    if (candidate1 == candidate2) {
        return result;
    }

    if (count(nums.begin(), nums.end(), candidate2) > n/3) {
        result.push_back(candidate2);
    }

    return result;
}
\end{verbatim}

\section{\(\lfloor\frac{n}{4}\rfloor\) \(\lfloor\frac{n}{k}\rfloor\)}
\label{sec:orgdfb8a23}

\href{https://stackoverflow.com/questions/24691048/find-all-elements-that-appear-more-than-n-4-times-in-linear-time}{Find all elements that appear more than n/4 times in linear time}

\href{https://stackoverflow.com/questions/3001181/find-if-there-is-an-element-repeating-itself-n-k-times/24642388\#24642388}{Find if there is an element repeating itself n/k times}
\end{document}